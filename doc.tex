% Created 2021-01-10 Sun 19:28
% Intended LaTeX compiler: pdflatex
\documentclass[11pt]{article}
\usepackage[utf8]{inputenc}
\usepackage[T1]{fontenc}
\usepackage{graphicx}
\usepackage{grffile}
\usepackage{longtable}
\usepackage{wrapfig}
\usepackage{rotating}
\usepackage[normalem]{ulem}
\usepackage{amsmath}
\usepackage{textcomp}
\usepackage{amssymb}
\usepackage{capt-of}
\usepackage{hyperref}
\usepackage{minted}
\usepackage[UTF8]{ctex}
\setCJKmainfont{宋体}
\usepackage[a4paper]{geometry}
\geometry{left=2.0cm,right=2.0cm,top=2.5cm,bottom=2.5cm}
\usepackage{fancyhdr}
\usepackage{nopageno}
\renewcommand{\baselinestretch}{1.0}
\setminted{tabsize=4,breaklines=true,frame=lines,framesep=2mm,fontsize=\scriptsize}
\usepackage[figurename=Fig.]{caption}
\newcounter{counter_exm}\setcounter{counter_exm}{1}
\usepackage{color}
\usepackage{listings}
\usepackage{caption}
\newcounter{nalg} % defines algorithm counter for chapter-level
\renewcommand{\thenalg}{\thechapter \arabic{nalg}} %defines appearance of the algorithm counter
\DeclareCaptionLabelFormat{algocaption}{Algorithm \thenalg} % defines a new caption label as Algorithm x.y
\lstnewenvironment{algorithm}[1][]
{
\refstepcounter{nalg} %increments algorithm number
\captionsetup{labelformat=algocaption,labelsep=colon} %defines the caption setup for: it ises label format as the declared caption label above and makes label and caption text to be separated by a ':'
\lstset{ %this is the stype
mathescape=true,
frame=tB,
numbers=left,
numberstyle=\tiny,
basicstyle=\small\ttfamily,
keywordstyle=\color{black}\bfseries\em,
keywords={,Input, Output, Return, Datatype, Function, In, If, Else, Foreach, For, While, Begin, End, Let, Swap, Sort, And, Or, Then, To,} %add the keywords you want, or load a language as Rubens explains in his comment above.
numbers=left,
xleftmargin=.04\textwidth,
#1, % this is to add specific settings to an usage of this environment (for instnce, the caption and referable label)
tabsize=4,
literate={\ \ }{{\ }}1
}
}
{}
\date{}
\title{}
\hypersetup{
 pdfauthor={Cothrax},
 pdftitle={},
 pdfkeywords={},
 pdfsubject={},
 pdfcreator={Emacs 26.3 (Org mode 9.1.9)}, 
 pdflang={English}}
\begin{document}

\pagestyle{fancy}
\lhead{\kaishu 中国科学院大学}
\chead{}
\rhead{\kaishu 2020年秋季学期~人工智能基础}

\begin{center}
  {\LARGE \bf 德州扑克AI报告}\\
\end{center}
\begin{flushright}
  { \heiti
    学号:2018K8009918009~2018K8009915034~2018KXXXXXXXXXX \\
    姓名:梁苏叁~李龙成~陈彦帆\\
    箱子:7 \\
  }
\end{flushright}


\section{问题介绍}
\label{sec:org5e5f5d8}
德州扑克属于不完美信息博弈的一种,一直被人工智能学界作为测试解决不完美
信息博弈的算法的基准游戏之一,多种基于CFR的方法已经成功解决了2人的德州
扑克AI问题,但6人无限注的德扑AI依然是一个open的问题。在本次大作业中,
我们结合的搜索和深度学习的方法,设计了6人无限注的德扑AI DeepFool,并在
\texttt{neuron\_poker} 环境对模型进行了测试。

\section{模型与算法}
\label{sec:org378e0a7}
DeepFool 采用了蒙特卡洛虚拟遗憾最小化(MCCFR)与神经网络结合的方法,在
一轮迭代中,用MCCFR与一个策略网络交互,计算遗憾值产生带标注的训练样本,
再交给策略网络学习,通过反复迭代收敛到一个较优的策略。在实际测试中,通
过将环境信息输入策略网络来得到决策的概率分布。

\subsection{MCCFR}
\label{sec:org8ffa411}
算法的主要框架采用了虚拟遗憾最小化(CFR)的方法,寻找一个理论上为纳什
均衡的博弈策略。我们将策略定义为一个已知信息 \(I\) 到合法行动的概率分布
\(\pi\) 的映射 \(\sigma:I\to\pi\) ,已知信息 \(I\) 包含了对于当前玩家的所有
可见信息,如手牌、公共牌、决策历史等,概率分布 \(\pi\) 的每一项 \(\pi_a\)
代表在已知 \(I\) 的情况下采取行动 \(a\) 的可能性。

CFR是一种求解纳什均衡策略的高效算法,它的一次迭代产生一颗博弈树,博弈
树上的所有玩家都根据 策略 \(\sigma\) 行动,每个节点 \(u\) 都会得到一个每个
玩家 \(p\) 的期望收益 \(u_{u,p}\) ,对于终止节点,收益即为游戏结束后赢得的
筹码;对于非终止节点,收益为该节点的所有子节点的收益的概率加权
(\ref{eq:cfr2})。树上的一个节点已知信息对应了映射 \(\sigma\) 中的一项 \(I\)
,在这个节点上一个合法行动 \(a\) 的遗憾值 (\ref{eq:cfr3}) 等于以概率1采取
行动 \(a\) 的期望收益减去 \(u\) (严格来讲,还需要乘上从根到该节点上,其他
玩家决策概率贡献的乘积作为系数),每次迭代遗憾值累加后再归一化就得到了
\(I\) 下的决策概率分布。CFR被证明可以在 \(O(N^2)\) ( \(N\) 状态空间大小)内
收敛到一个纳什均衡的策略 \cite{neller2013introduction} 。

\begin{align}
  \label{eq:cfr1} v &=  \textsc{Next-State}(u, a) \\
  \label{eq:cfr2} \textsc{Util}_{u,p} &= \sum_{a\in \textsc{Action}(u)} \pi_a\cdot \textsc{Util}_{v,p} \\
  \label{eq:cfr3} \textsc{Regret}_{u,a} &\propto \textsc{Util}_{v, p}-\textsc{Util}_{u,p}
\end{align}


但在6人的德州扑克中,直接使用朴素的CFR会带来两个问题:
\begin{enumerate}
\item 博弈树太大,比如在 \texttt{neuron\_poker} 环境中,环境实际只允许每轮加注1次,
这意味着最多有8轮迭代,环境允许的行动有6种,则一次完整的迭代需要访
问 \(6^{6\times 8}\) 个节点,这依照当代计算机的算力是不现实的。
\item 状态空间太大,即 \(\sigma\) 里的项数太多,需要大量的迭代才能收敛到一
个较好的策略。
\end{enumerate}

针对这两个问题,我们的模型对朴素的CFR做出了两点改进。
\begin{enumerate}
\item 对于第1个问题,我们采取的蒙特卡洛采样(MC)的方法,具体来讲,对于每
次迭代,固定一个当前玩家 \(P\) ,只在 \(P\) 决策的节点展开子树,在其余
玩家决策的节点直接根据决策概率采样,这样搜索的规模被缩小到了
\(6\times 6^6\) 。可以证明当采样次数足够多时,每个节点采样得到效用的期望
等于实际的效用值 \cite{burch2012efficient} 。
\item 对于第2个问题,我们用一个策略神经网络来拟合决策映射 \(\sigma\) ,替代
朴素CFR采用的表格法,CFR在搜索时通过询问策略网络来遍历博弈树并计算
遗憾值,用计算的遗憾值更新得到的节点的决策,将这些新决策作为产生的
训练样本交给策略网络学习。
\end{enumerate}

\subsection{}
\label{sec:orgca0f26b}
\subsection{策略网络}
\label{sec:orga45c8b3}
// 梁老师
\subsection{牌力计算}
\label{sec:org04e34d4}
// 陈老师

\section{实验细节}
\label{sec:org050bede}
\subsection{预训练策略网络}
\label{sec:org0ee1659}

\subsection{基于CFR的决策历史学习}
\label{sec:org3a66aba}
在得到一个初步可用的网络后,再采用MCCFR来生成决策样本,但因为使用MCCFR
时每一个节点都需要询问一次神经网络,这样生成样本的效率偏低,所以这一部
分分为两个子步骤:
\begin{enumerate}
\item 通过直接采样得到大量的样本进行训练:想得到一次迭代的搜索树结构,只
需要知道采样的节点选择了哪个行动,在这一个子步骤,MCCFR在采样时不询
问策略网络,而直接随机选择一个节点,这样可以在完成整个搜索树的遍历
之后,将所有要询问的样本交给策略网络,可以大大加快迭代速度。
\item 但子步骤1的做法违背了MCCFR的原理,在理论上是不收敛的,所以在直接采
样了一定轮数之后,还是采用标准的MCCFR进行搜索树的遍历,伪代码如下:

\begin{algorithm}[caption={MCCFR训练决策网络伪代码}, label={code:mccfr}]
Input: Current state $game$, sampling player $player$, strategy network $network$
Output: Utility for current state $game$

Function $\textsc{MCCFR}$($game, player, network$) Begin
	If $\textsc{Status}(game) = \textsc{GAME\_OVER}$ Then
		Return $\textsc{Payoff}(game)$

	$strategy \gets$ $\textsc{Get-Strategy}(network, game)$
	If $\textsc{Player}(game)=player$ Then Begin
		$action \gets \textsc{Sample-Action}(strategy)$
		$next \gets \textsc{Take-Action}(game, action)$
		Return $\textsc{MCCFR}(next, player, network)$
	End Else Begin
		$cvUtil \gets [0, 0, \cdots, 0]$
		$util \gets [0, 0,  \cdots, 0]$
		For $action \in$ $\textsc{Legal-Action}(game)$ Begin
			$next \gets \textsc{Take-Action}(game, action)$
			$nextUtil \gets \textsc{MCCFR}(next, player, network)$
			$cvUtil_{action} \gets nextUtil_{player}$
			$util \gets util + strategy_{action}\cdot nextUtil$
		End
		$regret \gets cvUtil - util_{player}$
		$\textsc{Update-Strategy}(network, regret)$
		Return $Util$
	End
End
\end{algorithm}
\end{enumerate}

\subsection{多模型集成}
\label{sec:org4acab7c}

\section{结果与讨论}
\label{sec:org9cbaef5}
\subsection{\texttt{neuron\_pocker} 环境测试结果}
\label{sec:org7b5b308}



\subsection{存在问题与讨论}
\label{sec:orgfb245b2}
\subsubsection{模型的收敛问题}
\label{sec:org9516eb9}
\subsubsection{\texttt{neuron\_poker} 环境的问题}
\label{sec:org32cde1b}
在测试中,我们发现 \texttt{neuron\_pocker} 实际存在如下问题:
\begin{enumerate}
\item 大作业的要求为设计6人无限注德州扑克AI,但环境中的
\texttt{is\_raising\_allowed} 函数每轮只允许加注6次,也就是说环境中游戏最多
进行8圈(每轮一圈加注,一圈跟注)
\item 环境中的历史信息没有严格的时序关系,只有在某一轮每个人是否
raise/check,也没有提供fold的决策历史和有多少人出局的信息,这给决策
历史部分网络的设计带来的很大的困扰。
\end{enumerate}

\nocite{moravcik17_deeps}

\bibliography{citations}
\bibliographystyle{plain}
\end{document}