% Created 2021-01-10 Sun 15:51
% Intended LaTeX compiler: pdflatex
\documentclass[11pt]{article}
\usepackage[utf8]{inputenc}
\usepackage[T1]{fontenc}
\usepackage{graphicx}
\usepackage{grffile}
\usepackage{longtable}
\usepackage{wrapfig}
\usepackage{rotating}
\usepackage[normalem]{ulem}
\usepackage{amsmath}
\usepackage{textcomp}
\usepackage{amssymb}
\usepackage{capt-of}
\usepackage{hyperref}
\usepackage{minted}
\usepackage[UTF8]{ctex}
\setCJKmainfont{宋体}
\usepackage[a4paper]{geometry}
\geometry{left=2.0cm,right=2.0cm,top=2.5cm,bottom=2.5cm}
\usepackage{fancyhdr}
\usepackage{nopageno}
\renewcommand{\baselinestretch}{1.0}
\setminted{tabsize=4,breaklines=true,frame=lines,framesep=2mm,fontsize=\small}
\date{}
\title{}
\hypersetup{
 pdfauthor={Cothrax},
 pdftitle={},
 pdfkeywords={},
 pdfsubject={},
 pdfcreator={Emacs 26.3 (Org mode 9.1.9)}, 
 pdflang={English}}
\begin{document}

\pagestyle{fancy}
\lhead{\kaishu 中国科学院大学}
\chead{}
\rhead{\kaishu 2020年秋季学期~计算机体系结构研讨课}

\begin{center}
  {\LARGE \bf 实验~13~报告}\\
\end{center}
\begin{flushright}
  { \heiti
    学号:2018K8009918009~2018K8009915034~2018KXXXXXXXXXX \\
    姓名:梁苏叁~李龙成~陈彦帆\\
    箱子:7 \\
  }
\end{flushright}


\section{问题介绍}
\label{sec:org585d0be}


\section{模型与算法}
\label{sec:org71115aa}
我们的德扑AI DeepFool 采用了蒙特卡洛虚拟遗憾最小化(MCCFR)与神经网络
结合的方法,在一轮迭代中,用MCCFR与一个策略网络交互,计算遗憾值产生带
标注的训练样本,再交给策略网络学习,通过反复迭代收敛到一个较优的策略。
在实际测试中,通过将环境信息输入策略网络来得到决策的概率分布。

\subsection{MCCFR}
\label{sec:org1094aa0}
\subsection{策略网络}
\label{sec:org8b0e6fc}
// 梁老师

\subsection{牌力计算}
\label{sec:orge742ac6}
// 陈老师

\section{实验细节}
\label{sec:orgb4e5d8d}
\subsection{预训练策略网络}
\label{sec:org9328ea4}
\subsection{基于CFR的决策历史学习}
\label{sec:org32c6ee0}
\subsection{多模型集成}
\label{sec:orga37efbc}
\section{结果与讨论}
\label{sec:org67e00d4}
\subsection{\texttt{neuron\_pocker} 环境测试结果}
\label{sec:org5620382}
\subsection{存在问题与讨论}
\label{sec:org5b913f4}
\subsubsection{模型的收敛问题}
\label{sec:orgaad24c7}
\subsubsection{}
\label{sec:org109b4fe}
\section{参考文献}
\label{sec:orgf73946a}
\end{document}